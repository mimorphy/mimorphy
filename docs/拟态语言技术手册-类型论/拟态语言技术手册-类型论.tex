\documentclass[12pt,a4paper,UTF8]{ctexart}
\usepackage{geometry}
\geometry{a4paper, margin=2.5cm}
\usepackage{hyperref}
\usepackage{amsmath, amssymb}
\usepackage{amsthm}
\usepackage{graphicx}
\usepackage{fancyhdr}
\usepackage{color}
\usepackage{xcolor}
\usepackage{listings}
\usepackage{titlesec}
\usepackage{tabularx}
\usepackage{colortbl}
\usepackage{multirow}
\usepackage{stmaryrd}

\lstset{
    basicstyle=\ttfamily,
    keywordstyle=\color{blue},
    extendedchars=false, % 关闭扩展字符支持
    backgroundcolor=\color{gray!5},   % 浅灰色背景
    frame=shadowbox,                  % 带阴影的边框
    framesep=5pt,                     % 边框内边距
    rulesepcolor=\color{gray!30},     % 边框颜色
    rulecolor=\color{blue!60},        % 阴影颜色
    numbers=left,                     % 行号在左侧
    numberstyle=\color{gray},    	  % 行号样式
    breaklines=true,                  % 自动换行
    captionpos=b,                     % 标题位置
    showstringspaces=false,           % 隐藏字符串中的空格
    tabsize=2,                        % 制表符宽度
    keywordstyle=\color{blue},        % 关键词颜色
    commentstyle=\color{olive},       % 注释颜色
    stringstyle=\color{red},          % 字符串颜色
    escapeinside=``,                  % 中文兼容设置
	keepspaces=true
}

% 定义浅灰色变量文本命令
\definecolor{lightgray}{gray}{0.6} % 灰度值0.7(0为黑,1为白)

\newcommand{\var}[1]{\textcolor{lightgray}{\mathtt{#1}}}

\newcommand{\concept}[1]{\text{引用: }\textcolor{blue}{#1}}

\newcommand{\prac}[1]{\textcolor{red}{#1}}

% 页眉页脚设置
\pagestyle{fancy}
\setlength{\headheight}{14.5pt}
\setlength{\parindent}{0pt}
\fancyhf{}
\fancyhead[L]{拟态语言技术手册}
\fancyhead[R]{\thepage}

% 代码高亮设置
\lstset{
    basicstyle=\ttfamily\small,
    keywordstyle=\color{blue},
    commentstyle=\color{green},
    stringstyle=\color{red},
    breaklines=true,
    numbers=left,
    numberstyle=\tiny,
    frame=single,
    backgroundcolor=\color[RGB]{245,245,244}
}

% 标题样式
\titleformat{\section}{\large\bfseries}{\thesection}{1em}{}
\titleformat{\subsection}{\normalsize\bfseries}{\thesubsection}{1em}{}

% 封面
\title{拟态语言技术手册\\{\small 类型论}}
\author{之恪提案}
\date{\today}

\begin{document}

\maketitle
\thispagestyle{empty} % 取消标题页的页码

\pagenumbering{Roman} % 使用大写罗马数字作为前置内容的页码
\setcounter{page}{0} % 重置页码为1

\maketitle
\clearpage

\tableofcontents
\newpage

\pagenumbering{arabic} % 切换到阿拉伯数字页码
\setcounter{page}{1} % 重置页码为1

\section{无类型的\texorpdfstring{$\lambda$}{}演算}

$\lambda$演算具有三种构造方式,变量、抽象、应用。

\subsection{变量}

\qquad 变量,即$\var{x},\var{y},\var{z}$等,一般要求变量的名称以字母或下划线开头,其余部分可以使用字母、下划线和数字。

\subsection{抽象}

\qquad 抽象,即$\lambda \var{x}. \; \var{M}$,其中$\var{x}$为变量,$\var{M}$为函数体。要求函数体不能为空,可以是其他的变量、抽象、应用。

\subsection{应用}

\qquad 应用,即$\var{M} \; \var{N}$,即将$\var{M}$应用于$\var{N}$。

\subsection{具体示例}

\qquad 定义:$\mathrm{TRUE} = \lambda \var{x}. \; \lambda \var{y}. \; \var{x}$,这个抽象输入参数$\var{x}$,返回另一个函数$\lambda \var{y}. \; \var{x}$(这个函数接受参数$\var{y}$,但忽略它,直接返回$\var{x}$)
因此,$\mathrm{TRUE}$的行为可以概括为:无论参数$\var{y}$为何值,永远返回参数$\var{x}$(选择第一个参数)。

\qquad 类似的,定义:$\mathrm{FALSE} = \lambda \var{x}. \; \lambda \var{y}. \; \var{y}$,无论参数$\var{x}$为何值,永远返回参数$\var{y}$(选择第二个参数)。

\qquad 我们可以尝试应用$\mathrm{TRUE}$,例如:
\begin{align*}
    &\mathrm{TRUE} \; \var{a} \; \var{b} \\
    &= (\lambda \var{x}. \; \lambda \var{y}. \; \var{x}) \; \var{a} \; \var{b} \\
    &= (\lambda \var{y}. \; \var{a}) \; \var{b} \\
    &= \var{a} \\
\end{align*}
应用的过程中的每一步,可以大体描述为:若有$\lambda \var{x}. \; \var{M} \; \var{y}$,则它等价于——一个去掉头部的$\lambda \var{x}.$和参数$\var{y}$,将$\var{M}$中的同一作用域的所有$\var{x}$替换为$\var{y}$的$\lambda$表达式。

\qquad 我们将使用“$\lambda$”和“$.$”包裹起来的变量称为绑定变量,例如$\lambda \var{x}.$中的$\var{x}$是一个绑定变量。
将没有使用“$\lambda$”和“$.$”包裹起来的变量称为自由变量,例如$\var{a}$是一个自由变量。
将一个$\lambda$表达式中的“$($”和“$)$”包括起来的内容称为该$\lambda$表达式的子表达式,它可以整体作为一个参数传递给绑定变量,也可以作为一个函数接收其他参数。

\qquad 例子:$\lambda \var{x}. \; \lambda \var{y}. \; \var{x} \; \var{y} \; \var{a} \; \var{b}$,
首先可以发现,第一个参数$\var{a}$将与$\var{x}$相绑定,因此去除$\lambda \var{x}.$和\\$\var{a}$,
并将$\lambda \var{y}. \; \var{x} \; \var{y} \; \var{b}$中的所有$\var{x}$替换为$\var{a}$,即$\lambda \var{y}. \; \var{a} \; \var{y} \; \var{b}$。
同理,进行第二次替换,可以发现第二个参数$\var{b}$将与$\var{y}$相绑定,因此去除$\lambda \var{y}.$和$\var{b}$,
并将$\var{a} \; \var{y}$中的所有$\var{y}$替换为$\var{b}$,就能得到结果$\var{a} \; \var{b}$。

\subsection{判断一个\texorpdfstring{$\lambda$}{}表达式是否合法}

判断一个$\lambda$表达式是否合法,可以关注以下几点特征:
\begin{itemize}
    \item 表达式中的变量是否使用了规定的命名规则?
    \item 每个$\lambda$抽象的后方是否都有函数体?
    \item 绑定变量的声明是否符合语法?
\end{itemize}

\subsection{\texorpdfstring{$\lambda$}表达式的定义的展开}

\qquad 由于可以使用标识符定义指代$\lambda$表达式,例如$\mathrm{TRUE} = \lambda \var{x}. \; \lambda \var{y}. \; \var{x}$。
因此,定义的展开可以理解为将定义的标识符替换为定义的$\lambda$表达式,但是替换时需要在左右分别加上左括号和右括号,表示这个子表达式是一个整体。

\qquad 例如,定义:$\mathrm{AND} = \lambda \var{p}. \; \lambda \var{q}. \; \var{p} \; \var{q} \; \mathrm{FALSE}$,
其等价于:$\lambda \var{p}. \; \lambda \var{q}. \; \var{p} \; \var{q} \; (\lambda \var{x}. \; \lambda \var{y}. \; \var{y})$。

\qquad 如果一个定义中包含其他定义,那么可以继续展开,直到只包含最基础的三种构造方式。

\qquad \textbf{注意:不允许两个定义形成递归的结构!}例如一个最简单的递归定义:$\mathrm{A} = \mathrm{B}, \mathrm{B} = \mathrm{A}$,
无论怎么展开,都无法将其展开为最基础的构造方式。并且会无限递归。

\subsection{作用域和变量遮蔽}

\qquad $\lambda$表达式的变量名可以重复,但是可能会产生变量遮蔽。
例如:$\lambda \var{x}. \; \lambda \var{x}. \; \var{x}$这个表达式,
外层的第一个绑定变量$\var{x}$被内层的第二个绑定变量$\var{x}$所遮蔽。
导致应用时,第一个绑定变量将无法影响函数体中的$\var{x}$,例如$\lambda \var{x}. \; \lambda \var{x}. \; \var{x} \; \var{a} \; \var{b} = \var{b}$。

\qquad 也可以认为外部的绑定变量与内部的绑定变量作用域不同,且内部绑定变量的作用域优先级更高。
另外,绑定在定义时确定,例如$\var{x} \; \lambda \var{x}.$,由于绑定变量$\var{x}$在自由变量$\var{x}$之后,
因此它们不是同一个变量,第一个是未被绑定的自由变量。

\subsection{判断两个\texorpdfstring{$\lambda$}{}表达式是否等价}

判断两个$\lambda$表达式是否等价,可以运用如下的方法:
\begin{itemize}
    \item 两个$\lambda$表达式的形式必须相同,如果把绑定变量、自由变量、子表达式视为基础的元素,那么它们排列的顺序必须一样,个数也要相同。
    \item 两个$\lambda$表达式中的所有相对应的子表达式必须等价,这两个$\lambda$表达式才可能等价。
    \item 两个$\lambda$表达式中位置相对应的绑定变量的名称可以不同,但它们的作用必须相同,即都在相对应的位置出现。
    \item 两个$\lambda$表达式中相对应的,且没有在相同作用域绑定过的自由变量必须同名。
\end{itemize}

\end{document}