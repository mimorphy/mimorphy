\documentclass[12pt,a4paper,UTF8]{ctexart}
\usepackage{geometry}
\geometry{a4paper, margin=2.5cm}
\usepackage{hyperref}
\usepackage{amsmath, amssymb}
\usepackage{amsthm}
\usepackage{graphicx}
\usepackage{fancyhdr}
\usepackage{color}
\usepackage{xcolor}
\usepackage{listings}
\usepackage{titlesec}
\usepackage{tabularx}
\usepackage{colortbl}
\usepackage{multirow}
\usepackage{stmaryrd}

\lstset{
    basicstyle=\ttfamily,
    keywordstyle=\color{blue},
    extendedchars=false, % 关闭扩展字符支持
    backgroundcolor=\color{gray!5},   % 浅灰色背景
    frame=shadowbox,                  % 带阴影的边框
    framesep=5pt,                     % 边框内边距
    rulesepcolor=\color{gray!30},     % 边框颜色
    rulecolor=\color{blue!60},        % 阴影颜色
    numbers=left,                     % 行号在左侧
    numberstyle=\color{gray},    	  % 行号样式
    breaklines=true,                  % 自动换行
    captionpos=b,                     % 标题位置
    showstringspaces=false,           % 隐藏字符串中的空格
    tabsize=2,                        % 制表符宽度
    keywordstyle=\color{blue},        % 关键词颜色
    commentstyle=\color{olive},       % 注释颜色
    stringstyle=\color{red},          % 字符串颜色
    escapeinside=``,                  % 中文兼容设置
	keepspaces=true
}

% 定义浅灰色变量文本命令
\definecolor{lightgray}{gray}{0.6} % 灰度值0.7(0为黑,1为白)

\newcommand{\var}[1]{\textcolor{lightgray}{\mathtt{#1}}}

\newcommand{\concept}[1]{\text{引用: }\textcolor{blue}{#1}}

\newcommand{\prac}[1]{\textcolor{red}{#1}}

% 页眉页脚设置
\pagestyle{fancy}
\setlength{\headheight}{14.5pt}
\setlength{\parindent}{0pt}
\fancyhf{}
\fancyhead[L]{拟态语言技术手册}
\fancyhead[R]{\thepage}

% 代码高亮设置
\lstset{
    basicstyle=\ttfamily\small,
    keywordstyle=\color{blue},
    commentstyle=\color{green},
    stringstyle=\color{red},
    breaklines=true,
    numbers=left,
    numberstyle=\tiny,
    frame=single,
    backgroundcolor=\color[RGB]{245,245,244}
}

% 标题样式
\titleformat{\section}{\large\bfseries}{\thesection}{1em}{}
\titleformat{\subsection}{\normalsize\bfseries}{\thesubsection}{1em}{}

% 封面
\title{拟态语言技术手册\\{\small 基础}}
\author{之恪提案}
\date{\today}

\begin{document}

\maketitle
\thispagestyle{empty} % 取消标题页的页码

\pagenumbering{Roman} % 使用大写罗马数字作为前置内容的页码
\setcounter{page}{0} % 重置页码为1

\maketitle
\clearpage

\tableofcontents
\newpage

\pagenumbering{arabic} % 切换到阿拉伯数字页码
\setcounter{page}{1} % 重置页码为1

\section{基础概念}

\subsection{位}

\qquad 位是计算机中表示信息的基本单位,在二进制计算机中,它可以表示2种状态(0和1),以此类推,在$\var{n}$进制中,它可以表示$\var{n}$种状态(0到$\var{n}-1$)。

\subsubsection{进制}

\qquad 进制(进位计数制)是人为定义的通过基数规定进位规则的计数方法,其核心特征为"逢基数进一"。例如:
\begin{itemize}
\item 十进制:基数为10,使用0-9共10个数码,逢十进一
\item 二进制:基数为2,使用0和1共2个数码,逢二进一
\item 八进制:基数为8,使用0-7共8个数码,逢八进一
\item 十六进制:基数为16,使用0-9和A-F共16个数码,逢十六进一
\end{itemize}

可以发现一个规律:$\var{n}$进制逢$\var{n}$进$1$,因此它的数码中不含$\var{n}$。例如十进制没有值为$10$的个位数,二进制没有值为$2$的个位数。

\subsubsection{基于进制的计数系统}

\qquad 在日常生活中,我们使用十进制计数。在十进制的计数系统中,我们把十进制数字的各个位置分为个位、十位、百位、千位...。
我们来解析一下,已知基数为$10$,假设一个三位数的个位数字为$\var{a}$,十位数字为$\var{b}$,百位数字为$\var{c}$。
那么它的值可以分解成$\var{a} + 10\var{b} + 100\var{c}$,也就是$1$倍的$\var{a}$、$10$倍的$\var{b}$、$10$倍的$10$倍的$\var{c}$求和。
换言之,即$10^0\var{a} + 10^1\var{b} + 10^2\var{c}$,其中可以发现一个规律:如果从$0$开始计数,那么第$0$位就是个位,权重为$10^0=1$;第$1$位就是十位,权重为$10^1=10$;第$2$位就是百位,权重为$10^2=100$。

\qquad 现在把视角换为二进制,二进制计数系统的基数为$2$,因此第$0$位的权重为$2^0=1$;第$1$位的权重为$2^1=2$;第$2$位的权重为$2^2=4$...。

\qquad 现在你应该可以理解,对于任意的$\var{n}$进制,它的计数系统是如何运作的了。

\subsubsection{存储空间中的计数系统}

\qquad 在二进制计算机的存储空间中,一个位可以表示$0$和$1$,可以发现它和二进制计数系统的位的表示范围相同,因此,我们可以用计算机中的位代表计数系统中的位。
例如,假设我们可以控制计算机中的$\var{n}$个位,那么我们就可以通过编码这$\var{n}$个位来表示一个$\var{n}$位的二进制数。
通常,这$\var{n}$个位在计算机中是顺序存储的,你可以把位想象为一个小方格,它们在存储空间中是紧挨着的,排成一个长条。

\qquad 现在你应该可以理解,计算机是如何表示数字的了。

\subsection{字节}

\qquad 字节是计算机中由多个位组成的一个单位,例如,在目前的计算机中,我们规定$8$个位组成一个字节。
假设进制为$\var{n}$,一个字节由$\var{m}$个位组成,那么一个字节的状态数为$\var{n}^{\var{m}}$,表示范围就是$[0, \var{n}^{\var{m}} - 1]$。

\subsection{地址}

\qquad 地址是一个数字,每个地址都代表了计算机中某一个字节的位置。例如,地址$\var{n}$和$\var{n}+1$指向两个在存储空间中相邻的字节。

\qquad 因此,如果我们要准确地在计算机中表示一个唯一的位的位置,可以使用这个位所在的字节的地址,以及它在这个字节中是第几个位来表示。

\subsection{字长}

\qquad 字长是衡量计算机性能的一个关键指标,它决定了计算机一次操作所能处理的数据量。
例如,$64$位且字节为$8$位的计算机的字长就是$8$个字节,它决定了这台计算机一次操作可以处理$8$字节的数据。

\qquad 因此,此计算机中大部分的操作至少都可以一次性处理$8$个字节,例如加减乘除等。

\section{数胞}

\textbf{开头:我们要解决什么问题?}

\qquad 在前面的章节中,我们学习了比特、字节等存储单元。我们发现,无论是哪种进制的计算机,其存储单元都可以被概括为两个核心属性:

\qquad 1. 该单元所能呈现的\textbf{所有可能状态的数量}($\var{n}$)。

\qquad 2. 该单元在某一时刻所处的\textbf{具体状态值}($\var{m}$)。

\qquad 为了用一种统一、简洁的数学语言来描述计算机中所有的存储空间,我们引入了“数胞”(Numerical Cell)这一抽象概念。它将帮助我们跳出二进制、十进制等具体进制的限制,从更高层面理解存储的本质。

\subsection{数胞的定义与表示}

一个数胞(NC)可以由一个二元组完整定义:
\[
NC \equiv (\var{n},\ \var{m})
\]
其中:
\begin{itemize}
    \item $\var{n}$:表示该数胞的\textbf{状态数}(Number of States)。它一般是一个大于1的自然数,决定了该存储单元能表示多少种不同的情况。它本质上对应了前文所述的“进制”概念。
    \item $\var{m}$:表示该数胞的\textbf{值}(Value)。它是该存储单元在当前时刻的具体状态,是一个自然数,且必须满足 $0 \leq \var{m} < \var{n}$。
\end{itemize}

\subsection{数胞的简写符号}

为了书写方便,我们定义了一套简写符号:
\begin{itemize}
    \item \textbf{$NC$}:数胞(Numerical Cell)的通用缩写。
    \item \textbf{$NC_{\var{n}}$}:表示所有状态数为$\var{n}$的同一类型数胞的集合。它强调了存储单元的“类型”或“容量”。
    \item \textbf{$NC_{\var{n}}(\var{m})$}:表示一个状态数为$\var{n}$,且当前值为$\var{m}$的特定数胞。这是最完整的表示形式,同时包含了存储单元的属性和当前状态。
\end{itemize}

\textbf{示例}:
\begin{itemize}
    \item $NC_2(1)$:表示一个状态数为2(即二进制)、当前值为1的数胞。这其实就是我们熟悉的一个比特(Bit),其值为1。
    \item $NC_{256}(65)$:表示一个状态数为256、当前值为65的数胞。这恰好可以表示一个值为65的字节(Byte),因为一个8位二进制字节有$2^8=256$种状态。
    \item $NC_{10}(7)$:表示一个状态数为10(即十进制)、当前值为7的数胞。这可以想象为一个十进制的基本存储单元。
\end{itemize}

\subsection{数胞的抽象意义}

“计算机中所有的存储空间都可以抽象为一个数胞。” 这句话的含义是:无论存储空间的实际物理实现如何,我们都可以用数胞的二元组模型$(\var{n},\ \var{m})$来刻画它。

\begin{itemize}
    \item \textbf{若一个存储单位可以表示$\var{N}$个状态}:这意味着它的“类型”是$NC_{\var{N}}$。
    \item \textbf{且其当前值为$\var{M}$}:这意味着它的当前状态是$\var{M}$,且$0 \leq \var{M} < \var{N}$。
    \item \textbf{那么它可以被数胞$NC_{\var{N}}(\var{M})$表示}:这个数胞的表示完全捕获了该存储单元的核心信息。
\end{itemize}

数胞的抽象威力在于其通用性。它不关心底层是二进制电路、三进制器件还是其他任何物理实现,它只关心逻辑上的状态数量和一个具体的状态值。这使得我们可以在统一的框架下讨论不同架构的计算机存储问题。

\subsection{状态数公式的再现}

一个由$\var{k}$个$NC_{\var{n}}$类型的数胞连续组成的存储空间,其总状态数正是我们熟悉的公式:
\[
\text{总状态数} = \var{n}^{\var{k}}
\]
这个公式解释了为什么一个由8个$NC_2$(比特)组成的字节($\var{n}=2, \var{k}=8$)有256种状态($2^8$),也解释了一个由2个$NC_{10}$(十进制单元)组成的存储空间有100种状态($10^2$)。

\end{document}